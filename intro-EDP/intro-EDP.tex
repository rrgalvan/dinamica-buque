\documentclass[9pt, dvipsnames]{beamer}
\usepackage{current-definitions}

% Resolver problemas con negritas en Beamer
% https://tex.stackexchange.com/questions/565069/beamer-bold-math-no-longer-working
\DeclareFontShape{OT1}{cmss}{b}{n}{<->ssub * cmss/bx/n}{}

\title{Introducción a las EDP para fluidos}
\author{Rafa Rodríguez Galván}

\AtBeginSection[]{
  \begin{frame}
  \vfill
  \centering
  \begin{beamercolorbox}[sep=8pt,center,shadow=true,rounded=true]{title}
    \usebeamerfont{title}\insertsectionhead\par%
  \end{beamercolorbox}
  \vfill
  \end{frame}
}

\usepackage{xcolor}
\newcommand{\matcolbox}[2][red]{%
  \setlength{\fboxrule}{1.66pt}%
  \fcolorbox{#1}{white}{$\!#2\!$}}
\newcommand{\altbox}[4][red]{%
  \only<#2>{\matcolbox[#1]{#4}}
  \only<#3>{\matcolbox[white]{#4}{}}
}
\newcommand{\nota}[3][red]{%
  \only<#2>{{\color{#1}$\bullet$} #3}}


%=======================================================
\begin{document}
%=======================================================

\begin{frame}
\maketitle
\end{frame}


\begin{frame}{Ecuaciones en Derivadas Parciales (EDP)}
  \structure{\textbf{Tipos de EDP}} (lineales y de orden $\le 2$):
  \begin{itemize}
  \item Ecuaciones \alert{Elípticas}
      \par\quad Ejemplo canónico: \textbf{ecuación de Possion}:
      \begin{align*}
        -\Delta u &= f \quad \only<2->{\structure{\text{ en  } \Omega}}
      \end{align*}
  \item Ecuaciones \alert{Parabólicas}
      \par\quad Ejemplo canónico: \textbf{ecuación del calor}:
      \begin{align*}
        u_t - \Delta u &= f \quad \only<2->{\structure{\text{ en  } \Omega}}
      \end{align*}
  \item Ecuaciones  \alert{Hiperbólicas}
    \par\quad Ejemplo canónico: \textbf{ecuación de transporte}:
      \begin{align*}
        u_t + \mathbf{v} \grad u = f,
      \end{align*}
      \quad donde $\vv$ es un dato (vector que transporta a $u$).
  \end{itemize}
  \medskip
  \uncover<2->{\alert{Condiciones de contorno} (adicionales sobre
    $\partial\Omega$, unicidad de solución)
    \smallskip
  \begin{itemize}
  \item
    \only<2-3>{ \alert{$u=\structure g$} \quad
          sobre $\partial\Omega$ \ (Cond. Contorno
        \textbf{Dirichlet}, \structure{$g$: dato})}
    \only<4-5>{\alert{$u=\structure 0$} \quad
          sobre $\partial\Omega$ \ (Cond. Contorno
        \textbf{Dirichlet \structure{Homogéneas}})}
  \item
    \only<3>{ \alert{$\grad u\cdot\nn = \structure g$} \quad
          sobre $\partial\Omega$ \ (Cond. Contorno
        \textbf{Neumann}, \structure{$g$: dato})}
    \only<5>{ \alert{$\grad u\cdot\nn =\structure 0$} \quad
          sobre $\partial\Omega$ \ (Cond. Contorno
        \textbf{Neumann \structure{Homogéneas}})}
  % \item
  %   \only<2>{\bullet\ & \color{NavyBlue} \grad u \cdot \nn
  %     {\color{NavyBlue}={\color{NavyBlue}g} \quad\text{ sobre
  %       }\partial\Omega} \ \text{\small (Cond. Contorno
  %       \textbf{Neumann})}} \only<3>{\bullet\ & \color{NavyBlue} \grad
  %     u \cdot \nn {\color{NavyBlue}=0 \quad\text{ sobre
  %       }\partial\Omega} \ \text{\small (Cond. contorno
  %       \textbf{Neumann} \color{NavyBlue}{Homogéneas}})}
  \end{itemize}
  }
\end{frame}

\begin{frame}{Métodos numéricos para resolución aproximada de EDP}
  \begin{itemize}
    \setlength\itemsep{0.9em}
  \item \textbf{Método de las Diferencias Finitas} (MDF)
    Indicado para ecuaciones \textit{Elípticas}, \textit{Parabólicas} e \textit{Hiperbólicas}
  \item \textbf{Método de los Volúmenes Finitos} (MVF)
    Indicado para ecuaciones \textit{Hiperbólicas}
  \item \textbf{Método de los Elementos Finitos} (MEF)
    Indicado para ecuaciones \textit{Elípticas} y \textit{Parabólicas}
  \end{itemize}
\end{frame}

\begin{frame}{Ecuaciones de Navier-Stokes (I)}
  \begin{itemize}
  \item Sea $\Omega$ un dominio de $\Rset^d$, $d=3$. Puntos $(x,y,z)\in\Rset$
  \item<2-> \emph{Incógnitas}:
    \begin{itemize}
    \item $\uu=(u,v,w)$: campo de velocidades. $u$, $v$ y $w$: $\Omega \to \Rset$
    \item $p$: presión, $p:\Omega \to \Rset$
    \end{itemize}
  \item <2->
    \begin{small}
      Cada incógnita es una función, que varía en cada punto
      $(x,y,z)\in\Omega$ \par\hfill \dots y usualmente, también con el tiempo, $t$. Por ejemplo, $p=p(x,y,z,t)$
    \end{small}
  \end{itemize}
  \begin{center}
    \begin{overprint}
      \includegraphics<1>[width=0.75\linewidth]{domain-cylinder.png}
      \includegraphics<2->[width=0.75\linewidth]{v+p-cylinder.png}
    \end{overprint}
  \end{center}
\end{frame}

\begin{frame}{Ecuaciones de Navier-Stokes (II)}
  \alert{Expresión abreviada} (luego la desarrollaremos):
  \begin{align*}
    \altbox[ForestGreen]{2}{-1,3-}{
    \uu_t
    + \uu\cdot\grad \uu
    - \nu \Delta \uu
    +  \grad p
    = \ff \text{ en } \Omega,
    }
    \\
    \altbox[pink]{3}{-2,4-}{
    \div{\uu} = 0 \text{ en } \Omega.}
  \end{align*}
  \only<2>{{\color{ForestGreen}$\bullet$} Ecuación de momentos (vectorial, pues $\uu=(u,v,w)$)}
  \only<3>{{\color{pink}$\bullet$} Ecuación de incompresibilidad  o de conservación de la masa}
  \only<4->{
  \emph{Datos}:
  \begin{itemize}
    \setlength\itemsep{0.9em}
  \item<4-> $\nu>0$: coeficiente de \structure{viscosidad} (cinemática)
    \par\hfill ``inverso del número de \structure{Reynolds}'' ($Re = U \cdot L/\nu$)\footnote{
      Denotamos $U$ y $L$: velocidad y longitud características.}
  \smallskip
      \begin{itemize}
        \setlength\itemsep{0.8em}
      \item Caso $Re \ll$ (o bien $\nu\gg$ respecto a $U$, $L$): \textbf{flujo laminar}
      \item Caso $Re \gg$ (o bien $\nu\ll$ respecto a $U$, $L$): \textbf{flujo turbulento}
      \end{itemize}
    \item<4-> $\ff=(f_1,f_2,f_3)$ fuerza externa que actúa sobre el fluido (ejemplo: gravedad)
  \end{itemize}
  }
\end{frame}


\begin{frame}{Ecuaciones de Navier-Stokes (III)}
\begin{itemize}
\item \emph{Navier-Stokes, expresión extendida}:
  \begin{align*}
    \altbox[ForestGreen]{2}{-1,3-}{u_t}
    + \altbox{3}{-2,4-}{\uu\cdot\grad u}
    - \nu \altbox[NavyBlue]{4}{-3,5-}{\Delta u}
    + \altbox[orange]{5}{-4,6-}{\dx p}
    &= f_1 \text{ en } \Omega,
    \\
    \altbox[ForestGreen]{2}{-1,3-}{v_t}
    + \altbox{3}{-2,4-}{\uu\cdot\grad v}
    - \nu \altbox[NavyBlue]{4}{-3,5-}{\Delta v}
    + \altbox[orange]{5}{-4,6-}{\dy p}
    &= f_2 \text{ en } \Omega,
    \\
    \altbox[ForestGreen]{2}{-1,3-}{w_t}
    + \altbox{3}{-2,4-}{\uu\cdot\grad w}
    - \nu \altbox[NavyBlue]{4}{-3,5-}{\Delta w}
    + \altbox[orange]{5}{-4,6-}{\dz p}
    &= f_3 \text{ en } \Omega,
    \\
    \altbox[pink]{6}{-5}{\div{\uu}} &= 0 \text{ en } \Omega.
  \end{align*}
  \nota[ForestGreen]{2}{Variación del flujo con el tiempo, $t$}
  \nota{3}{Términos de inercia \texttt{no lineales} y ``de tipo \textbf{hiperbólico}}
  \nota[NavyBlue]{4}{Términos viscosos o difusivos}
  \nota[orange]{5}{Gradiente de presión}
  \nota[pink]{6}{Divergencia nula \scriptsize (incompresibilidad)}

%   Los términos más "delicados" (de convección o transporte no lineal)
%   \begin{align*}
%   \uu\grad u &= (u,v,w) \cdot (\dx u, \dy u, \dz u)^T \\
%   &= u \dx u + v \dy u + w \dz u
%   \\
%   \uu\grad v &= u \dx v + v \dy v + w \dz v
%   \\
%   \uu\grad w &= u \dx w + v \dy w + w \dz w
% \end{align*}
  \end{itemize}
  \vfill
\end{frame}


%=======================================================
\section{Formulación variacional de problemas elípticos}
%=======================================================

\begin{frame}{El problema de Poisson (I)}
  \begin{itemize}
    \setlength\itemsep{0.9em}
  \item Nos fijamos en los términos ``más sencillos'': los de tipo difusivo, $-\Delta u$
  \item A partir de ellos, iremos añadiendo términos más complejos
  \item El objetivo final:  reconstruir las ecuaciones de Navier-Stokes
  \end{itemize}
\end{frame}

\begin{frame}{El problema de Poisson (II)}
  Dado un conjunto abierto $\Omega$ (cuya frontera $\partial\Omega$ la
  podemos dividir en ``una parte Dirichlet'', $\Gamma_D$, y otra
  ``parte Neumann'', $\Gamma_N$, alguna de las cuales podría ser
  vacía), planteamos:
  \begin{equation*}
    \left\{
    \begin{aligned}
      -\Delta u &= f \text{ en } \Omega, \\
      u&=g_D \text{ sobre } \Gamma_D, \\
      \grad u \cdot \nn &=g_N \text{ sobre } \Gamma_N.
    \end{aligned}
    \right.
  \end{equation*}
  Para escribir la formulación variacional, definimos
  \begin{itemize}
  \item Un espacio vectorial $V$ (espacio para funciones test)
    $$ V = \{ v\in H^1(\Omega) \ / \ v=0 \text{ sobre } {\Gamma_D} \}$$
    \begin{flushright}\small
    En $V$ se recogen las posibles condiciones de contorno
      Dirichlet que aparezcan en el problema, pero de forma homogénea
      $(=0)$:
    \end{flushright}
\item Un espacio afín $V_{S}$ (espacio donde buscar la solución del problema)
    $$ V_S = \{ v\in H^1(\Omega) \ / \ v=g_D \text{ sobre } {\Gamma_D} \}$$
    \begin{flushright}\small
      En él se recogen las condiciones de contorno Dirichlet que
      aparezcan en el problema original
      \\
      Si no hay condiciones de contorno Dirichlet (o si $g_D=0$), este
      espacio será igual al de las funciones test ($V_S=V$)
    \end{flushright}
  \end{itemize}
\end{frame}

\begin{frame}{El problema de Poisson (III)}
  Y así escribimos la formulación variacional hallar $u\in V_S$ tal
  que, para todo $v\in V$:
  \begin{align*}
      \int_\Omega \grad u \cdot \grad v
      &- \int_{\partial\Omega} (\grad u \cdot \nn) v
      \\
      &=\int_\Omega \grad u \cdot \grad v
      - \int_{\Gamma_D} (\grad u \cdot \nn) v
      - \int_{\Gamma_N} (\grad u \cdot \nn) v
      \\
      &=\int_\Omega \grad u \cdot \grad v
      - \int_{\Gamma_N} g_N v
      = \int_\Omega  f v
    \end{align*}
    O bien:
    $$
    a(u,v) = L(v),
    $$
    donde
    \begin{equation*}
      a(u,v)=\int_\Omega \grad u \cdot \grad v,
      \quad
      L(v)=  \int_{\Gamma_N} g_N v + \int_\Omega  f v
    \end{equation*}
\end{frame}

\begin{frame}{Flujos potenciales (irrotacionales) e incompresibles}
\begin{itemize}
\item Suponemos que existe una función potencial, $\Phi$, tal que
  el flujo se puede obtener como el gradiente de este potencial, es decir
  $$u=\Psi_x, \ v=\Psi_y$$.
  \item Entonces, el flujo es \alert{irrotacional}:
  $$
  rot(u,v,w) = 0\cdot \vec{i} + 0\cdot \vec j + (v_x - u_y)\cdot \vec k = 0,
  $$
  pues
  $$
  v_x - u_y = (\Psi_{y})_x - (\Psi_{x})_y = \Psi_{yx} - \Psi_{xy} = \Psi_{xy} - \Psi_{xy} = 0.
  $$
\item Si además el potencial es solución de un problema de Laplace, del tipo
  \begin{align}
    \label{potencial.eq.1}
    -\Delta \Psi &= 0 \quad \text{ en } \Omega, \\
    \label{potencial.eq.2}
    &+ \text{ cond. de contorno},
  \end{align}
  entonces el flujo es \alert{incompresible}:
  $$
  \nabla\cdot(u,v) = u_x + v_y =\Psi_{xx} + \Psi_{yy} =  0.
  $$
\item Así, para determinar un flujo irrotacional e incompresible en
  cualquier dominio $\Omega\subset\Rset^2$, basta proceder como sigue:
  \begin{enumerate}
  \item Resolver el
    problema~(\ref{potencial.eq.1})--(\ref{potencial.eq.2}), por
    ejemplo usando el método de los elementos finitos
  \item Tomar el flujo $u=\Psi_x$, $v=\Psi_y$.
  \end{enumerate}


\end{itemize}

\end{frame}

%=======================================================
\section{Formulación variacional de problemas parabólicos}
%=======================================================

%------------------------------------------------------
\begin{frame}{Ecuación del calor}
%------------------------------------------------------
  \begin{itemize}

  \item
  Fijamos un dominio espacial $\Omega\subset\Rset^n$,
  con frontera $\partial\Omega=\Gamma_D\cup\Gamma_N$ y un \textbf{intervalo de
    tiempo} $[0,T]$.


\item
  Planteamos: hallar $u=u(\mathbf{x}, t)$, siendo $\mathbf{x}=(x,y)$ en el caso 2D
  o $\mathbf{x}=(x,y,z)$ en 3D, solución del problema:
   \begin{align*}
   \dt u - \nu\Delta u &= f, \quad \forall(\mathbf{x},t)\text{ de }\Omega\times (0,T), \\
   u &= g_D \quad\text{ sobre } \Gamma_D \quad \text{(c. de contorno Dirichlet)},\\
   \grad u\cdot\nn &= g_N \quad\text{ sobre } \Gamma_D \quad \text{(c. de contorno Neumann)},\\
   u(\mathbf{x},t=0) &= u_0(\mathbf{x}) \quad \text{(condición inicial)}.
 \end{align*}
 donde $f(\mathbf{x},t)$, $g_D(\mathbf{x},t)$, $g_N(\mathbf{x},t)$ y
 $u_0(\mathbf{x})$ son funciones conocidas (datos). $\nu>0$ es también un dato (coeficiente de difusión).

 \vfill
 \begin{small}
 \begin{itemize}
 \item
   En la ecuación anterior estamos de nuevo dividiendo la frontera de
   omega en dos zonas: $\partial\Omega=\Gamma_D\cup\Gamma_N$ (para
   permitir la posibilidad de que se impongan tanto condiciones de
   contorno de tipo Dirichlet como de tipo Neumann).
 \item Existiría la posibilidad de que $\Gamma_D=\emptyset$ (y por
   tanto sólo imponemos condiciones Neumann en
   $\Gamma_N=\partial\Omega$)) o $\Gamma_N=\emptyset$ (y por tanto
   $\Gamma_D$ es toda la frontera y sólo imponemos condiciones
   Dirichlet)
 \end{itemize}
\end{small}
\end{itemize}
\end{frame}


%------------------------------------------------------
\begin{frame}{Discretización en tiempo}
%------------------------------------------------------
  \small
Consideramos la siguiente partición del \textbf{intervalo de tiempo} $[0,T]$:

$$
0=t_0 < t_1 < t_2 < \dots < t_m < t_{m+1} < \dots  < t_M=T.
$$
\begin{itemize}
  \item Suponemos que $t_{m+1}-t_m = \Delta t$ (constante).
  \item Intentamos hallar aproximaciones de la solución $u^m\approx u(t_m)$ ($m=0,1,...,M$).
\end{itemize}
Calculamos los valores $u^m$ de la siguiente forma:
\begin{itemize}
  \item Inicialización: para $m=0$: $u^0=u_0$ ($u_0$ es el dato inicial).
  \item Iteración de tiempo $m=1,2,\dots,M$: conocida la solución en
    la etapa $m-1$, hallamos $u^m$ como solución de:
    \begin{align}
      \label{eq:1}
     \frac{u^m-u^{m-1}}{\Delta t} - \nu\Delta u^m &= f^m \quad\text{ en } \Omega
     \\
     u^m &= g_D^m \quad\text{ sobre } \partial\Gamma_D
           \\
   \grad u^m\cdot\nn &= g_N^m \quad\text{ sobre } \partial\Gamma_D
   \end{align}
   donde $f^m(x)$ y $g^m(x)$ denotan respectivamente a $f(x,t^m)$ y $g(x,t^m)$.
   \item A este tipo de discretizaciones en tiempo se les llama método de \textbf{Euler implícito}.
   \item El problema anterior es estacionario (el tiempo $t_m$ está
     fijo) y lo podemos resolver mediante el MEF como veremos a
     continuación...

\end{itemize}

\end{frame}

%------------------------------------------------------
\begin{frame}{Formulación variacional de la Ecuación del Calor}
%------------------------------------------------------
  \small
  En~(\ref{eq:1}) podemos pasar los datos al segundo miembro, obteniendo:
  \begin{equation}
    \label{eq:2}
     \frac{u^m}{\Delta t} - \nu\Delta u^m = \frac{u^{m-1}}{\Delta t} + f^m \quad\text{ en } \Omega
  \end{equation}
  Como antes, definimos los espacios de funciones:
  \begin{itemize}
  \item Un espacio vectorial $V$ (espacio para funciones test)
    $$ V = \{ v\in H^1(\Omega) \ / \ v=0 \text{ sobre } {\Gamma_D} \}$$
\item Un espacio afín $V_{S}$ (espacio donde buscar la solución del problema)
    $$ V_S = \{ v\in H^1(\Omega) \ / \ v=g_D \text{ sobre } {\Gamma_D} \}$$
  \end{itemize}
  Multiplicando por funciones test $\bu\in V$ en~(\ref{eq:2}) e
  integrando por partes, deducimos que $u^m$ verifica:
  \begin{align*}
    \int_\Omega \frac{u^m}{\Delta t} \bu + \int_\Omega \nu\grad u^m \grad\bu
    - \int_{\partial\Omega} \grad u \cdot \nn \,\bu
    = \int_\Omega \frac{u^{m-1}}{\Delta t} \bu + \int_\Omega f^m  \bu
  \end{align*}
  Y usando (en la integral frontera) que $\bu=0$ sobre $\Gamma_D$ y $\nabla u^m\cdot\nn=g_n$ sobre
  $\Gamma_N$ llegamos a la formulación variacional para el instante $m$:
  \begin{align*}
    \structure{
    \int_\Omega \frac{u^m}{\Delta t} \bu + \int_\Omega \nu\grad u^m \grad\bu
    = \int_\Omega \frac{u^{m-1}}{\Delta t} \bu + \int_\Omega f^m  \bu
    + \int_{\partial\Omega} g^m_N \,\bu
    }
  \end{align*}
\end{frame}



%=======================================================
\section{Formulación variacional de las ecuaciones de Stokes}
%=======================================================


\begin{frame}{Ecuaciones de Stokes estacionarias}
  \alert{Problema de Stokes de estacionario} (con condiciones de contorno \alert{Dirichlet}):

  \begin{itemize}


  \item Supongamos dado un dominio (un conjunto abierto con frontera
    regular) $\Omega\subset \Rset^3$.
  \item Consideramos los datos:
    \begin{align*}
      &\nu>0 \ \text{(coeficiente de viscosidad)},\\
      &\ff=(f_1,f_2,f_3):\Omega \to \Rset^3 \ \text{(fuerza externa)}
    \end{align*}

  \item Planteamos: hallar $u=u(\xx)$, $v=v(\xx)$, $w=w(\xx)$
    $p=p(\xx)$, todas ellas funciones definidas en un conjunto
    abierto $\Omega$ y con valores en $\Rset$, tales que:
    \begin{align*}
      &- \nu\Delta u + \dx p = f_1
      \\
      &- \nu\Delta v + \dy p = f_2
      \\
      &- \nu\Delta w + \dz p = f_3
      \\
      \div{\uu} &= \dx u + \dy v + \dz w = 0,
    \end{align*}
    junto a las condiciones de contorno (para los datos frontera $g_1$, $g_2$ y $g_3$)
    \begin{align*}
      u(\xx,t)=g_1(\xx),
      \quad
      v(\xx,t)=g_2(\xx),
      \quad
      w(\xx,t)=g_3(\xx),
      \qquad
    \forall \xx\in\partial\Omega.
    \end{align*}
  \end{itemize}

\end{frame}


\begin{frame}{Ecuaciones de Stokes Estacionarias: Formulación variacional (I)}
  \begin{itemize}

  \item Por simplicidad, consideramos condiciones Dirichlet homogéneas en toda la frontera de $\Omega$.
  \item Definimos los espacio $V=H^1_0(\Omega)$ y
    $Q=L_0^2(\Omega)$. Multiplicamos, respectivamente, a las tres
    ecuaciones de momentos en el problema de Stokes por funciones test
    $\bu, \bv, \bw \in V$. Multiplicamos la ecuación de divergencia
    por otra función test $q\in Q$.
  \item Integramos por partes cada una de estas ecuaciones, teniendo
    en cuenta que los términos frontera son cero y, por tanto,
    desaparecen. Así obtenemos:
    \begin{align*}
      \int_\Omega \nu\grad u\grad\bu - \int_\Omega p\dx \bu &= \int_\Omega f_1 \bu, \quad \forall \bu\in V,
      \\[0.5em]
      \int_\Omega \nu\grad v\grad\bv - \int_\Omega p\dy \bv &= \int_\Omega f_2 \bv, \quad \forall \bv\in V,
      \\[0.5em]
      \int_\Omega \nu\grad w\grad\bw - \int_\Omega p\dz \bw &= \int_\Omega f_3 \bw, \quad \forall \bw\in V,
      \\[0.5em]
      \int_\Omega \bp \div{\uu} & =  0, \quad \forall \bp \in Q.
    \end{align*}

  \end{itemize}
\end{frame}



\begin{frame}{Ecuaciones de Stokes Estacionarias: Formulación variacional (II)}
  \begin{itemize}
  \item Sumamos las ecuaciones anteriores y planteamos el problema:
    Hallar $u,v,w\in H^1_0(\Omega)$ y $p\in L_0^2(\Omega)$ tales que
    \begin{align*}
      \int_\Omega \nu\Big( \grad u\grad\bu &+
      \grad v\grad\bv + \grad w\grad\bw \Big)
      \\
      &- \int_\Omega \left( \dx \bu + \dy \bv + \dz \bw \right) p
      \\
      &+ \int_\Omega \left( \dx u + \dy v + \dz w \right) \bp
      \\
      &= \int_\Omega \left( f_1 \bu + f_2 \bv + f_3 \bw \right)
    \end{align*}
    $\forall \bu,\bv, \bw\in V$, $\forall\bp\in Q$.
  \end{itemize}

\end{frame}



%------------------------------------------------------
\begin{frame}{Ecuaciones de Stokes de evolución}
%------------------------------------------------------
  \small
  \alert{Problema de Stokes de evolución} (con condiciones de contorno \alert{Dirichlet}):

  \begin{itemize}
  \item Supongamos dado un dominio (un conjunto abierto con frontera
    regular) $\Omega\subset \Rset^3$ y un intervalo temporal $[0,T]$,
    siendo $T>0$ el tiempo final de observación.
  \item Consideramos los datos:
    \begin{align}
      &\nu>0 \ \text{(coeficiente de viscosidad)},\\
      &\ff=(f_1,f_2,f_3):\Omega\times [0,T] \to \Rset^3 \ \text{(fuerza externa)}
    \end{align}

  \item Planteamos: hallar $u=u(\xx,t)$, $v=v(\xx,t)$, $w=w(\xx,t)$
    $p=p(\xx,t)$, todas ellas funciones definidas en un conjunto
    abierto $\Omega$ y con valores en $\Rset$, tales que:
    \begin{align}
      \dt u &- \nu\Delta u + \dx p = f_1
      \\
      \dt v &- \nu\Delta v + \dy p = f_2
      \\
      \dz w &- \nu\Delta w + \dz p = f_3
      \\
      \div{\uu} &= \dx u + \dy v + \dz w = 0,
    \end{align}
    junto a las condiciones de contorno (para los datos frontera $g_1$, $g_2$ y $g_3$)
    \begin{align}
      u(\xx,t)=g_1(\xx,t),
      \quad
      v(\xx,t)=g_2(\xx,t),
      \quad
      w(\xx,t)=g_3(\xx,t),
      \\
    \forall \xx\in\partial\Omega, \ \forall t\in(0,T)
    \end{align}
    y las condiciones iniciales (para los datos iniciales $u_1$, $u_2$, $u_3$)
    $$
    u(\xx,0) = u_0(\xx), \quad v(\xx,0)=v_0(\xx), \quad w(\xx,0)=w_0(\xx),
    $$
  \end{itemize}

\end{frame}

%------------------------------------------------------
\begin{frame}{Discrtetización del problema de Stokes evolutivo}
%------------------------------------------------------

  \begin{itemize}

  \item Planteamos: hallar $u^m\approx u(\xx,t_m)$, $v^m\approx v(\xx,t_m)$, $w^m\approx w(\xx,t_tm)$, $p^m\approx p(\xx,t_m)$, todas ellas funciones definidas en un conjunto
    abierto $\Omega$, tales que:
    \begin{align}
      \frac{u^m - u^{m-1}}{\Delta t} &- \nu\Delta u^m + \dx p^m = f^m_1
      \\
      \frac{v^m - v^{m-1}}{\Delta t} &- \nu\Delta v^m + \dy p^m = f^m_2
      \\
      \frac{w^m - w^{m-1}}{\Delta t} &- \nu\Delta w^m + \dz p^m = f^m_3
      \\
      \div{\uu^m} &= \dx u^m + \dy v^m + \dz w^m = 0,
    \end{align}
    junto a las condiciones de contorno (para los datos frontera $g_1$, $g_2$ y $g_3$)
    \begin{align}
      u^m=g_1^m
      \quad
      v^m(\xx,t)=g_2^m
      \quad
      w^m(\xx,t)=g_3^m
      \\
    \forall \xx\in\partial\Omega
    \end{align}
  \end{itemize}

\end{frame}

\begin{frame}{Ecuaciones de Stokes Evolutivas: Formulación variacional (II)}
  En la etapa de tiempo $m$ (conocidas $u^{m-1}$, $v^{m-1}$, $w^{m-1}$), hallar $u^m$, $v^m$, $w^m \in H^1_0(\Omega)$ y $p^m\in L_0^2(\Omega)$ tales que:
    \begin{align*}
      \frac{1}{\Delta t} \int_\Omega\big(
      u^m \bu
      &+ {v^m} \bv
      + {w^m} \bw
      \big)
      + \int_\Omega \nu\Big( \grad u^m \grad\bu +
      \grad v^m\grad\bv + \grad w^m\grad\bw \Big)
      \\
      &- \int_\Omega \left( \dx \bu + \dy \bv + \dz \bw \right) p^m
      \\
      &+ \int_\Omega \left( \dx u^m + \dy v^m  + \dz w^m \right) \bp
      \\
      &=
      \frac{1}{\Delta t} \int_\Omega\big(
      u^{m-1} \bu
      + {v^{m-1}} \bv
      + {w^{m-1}} \bw
      \big)
       + \int_\Omega \left( f^m_1 \bu + f^m_2 \bv + f^m_3 \bw \right)
    \end{align*}
    $\forall \bu,\bv, \bw\in V$, $\forall\bp\in Q$.

\end{frame}

\end{document}

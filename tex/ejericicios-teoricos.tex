\documentclass{article}

\usepackage[spanish]{babel}
\usepackage[utf8]{inputenc}
% \usepackage[T1]{fontenc}
\usepackage[margin={2cm,1.5cm}]{geometry}
\usepackage{amsmath}
\usepackage{amssymb}

\newcommand{\Rset}{\mathbb{R}}
\newcommand{\uu}{\vec{\mathbf{u}}}
\newcommand{\ff}{\vec{\mathbf{f}}}
\newcommand{\dt}[1]{\frac{\partial #1}{\partial t}}
\newcommand{\dx}[1]{\frac{\partial #1}{\partial x}}
\newcommand{\dy}[1]{\frac{\partial #1}{\partial y}}
\newcommand{\laplace}{\nabla^2}
\newcommand{\grad}{\vec{\mathbf{\nabla}}}
\renewcommand{\div}{\nabla\cdot}

\begin{document}
\begin{titlepage}
  \begin{center}
    Dinámica del Buque (CFD) \hfill Máster en Ingeniería Naval \hfill
    Curso 2016/17
    \\[2em]
    {\LARGE Ejercicios Teóricos}
\end{center}

\bigskip

\begin{enumerate}
\item Sea $\Omega\subset \Rset^2$ el dominio comprendido entre un
  círculo $C$ y un sólido contenido en $C$ con superficie
  $S$. Consideremos un flujo potencial
  $$
  \uu(x,y)=
  \left(
    \begin{array}{r}
      \dy{\psi(x,y)} \\ [0.5em]
      -\dx{\psi(x,y)}
    \end{array}
  \right),
  $$
  donde $\psi$ es solución de:
  \begin{equation*}
   \left\{
     \begin{aligned}
       -\laplace\psi &=0 \quad \text{en } \Omega ,\\
       \psi|_S &=0, \quad \psi|_C= y.
     \end{aligned}
   \right.
 \end{equation*}
 Determinar la correspondiente formulación variacional.

\item %% 2
  Determinar la formulación variacional (mixta) para el problema de
  Stokes estacionario: hallar $\uu:\Omega \to \Rset^2$ y
  $p:\Omega\to\Rset$ tales que
  \begin{equation*}
   \left\{
     \begin{aligned}
       -\nu\laplace\uu + \grad p &= \ff \quad \text{en } \Omega ,\\
       \div\uu &=0 \quad\text{en } \Omega,\\
       \uu &=0 \quad\text{sobre } \partial\Omega,\\
     \end{aligned}
   \right.
 \end{equation*}
 donde $\Omega\subset\Rset^2$, $\nu\in\Rset$, $\nu>0$,
 $\ff:\Omega \to \Rset^2$ son datos conocidos.
\item %% 2
  Determinar la formulación variacional (mixta) para el siguiente
  problema de Stokes evolutivo: hallar $\uu:\Omega\times(0,T) \to \Rset^2$ y
  $p:\Omega\times(0,T)\to\Rset$ tales que
  \begin{equation*}
   \left\{
     \begin{aligned}
       \dt\uu-\nu\laplace\uu + \grad p &= \ff \quad \text{en } \Omega\times(0,T) ,\\
       \div\uu &=0 \quad\text{en } \Omega\times(0,T),\\
       \uu &=0 \quad\text{sobre } \partial\Omega, \quad \uu|_{t=0}=\uu_0\quad \text{en } \Omega
     \end{aligned}
   \right.
 \end{equation*}
 donde $\Omega\subset\Rset^2$, $T\in\Rset$ ($T>0$), $\nu\in\Rset$,
 ($\nu>0$), $\ff:\Omega \to \Rset^2$, $\uu_0:\Omega \to \Rset^2$ son
 datos conocidos. Usar, para aproximar la derivada temporal,
 diferencias finitas retrógradas:
 $$
 \dt{u}|_{t=t_{n+1}} \approx \frac{u^{n+1}-u^n}{\Delta t},
 $$
 donde $\Delta t>0$ es el paso en tiempo y $t_n=n\Delta t$, $n\ge 0$
 son instantes de tiempo en $[0,T]$.
\end{enumerate}



\end{titlepage}

\end{document}

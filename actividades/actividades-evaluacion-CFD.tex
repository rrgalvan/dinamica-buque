\documentclass[11pt]{article}

\usepackage[spanish]{babel}
\usepackage[utf8]{inputenc}
\usepackage{amsmath, amsfonts}
\usepackage[hmargin=1.5cm, vmargin=2cm]{geometry}
\usepackage{graphicx}
\usepackage{titling}

\usepackage{../intro-EDP/current-definitions}

%-------------------------------------------------------------
\setlength{\droptitle}{-5em}   % Espacio antes del título
\renewcommand{\maketitlehookd}{\vspace*{-5em}} % Ejecutado después del título

\newcommand{\cabecera}{\begin{center}\it
    \large
    Máster en Ingeniería Naval y Oceánica, curso 2019/2020
    \\[0.66em]
    {\Large\bf Dinámica del Buque: CFD}
    \\[0.66em]
    {\bf Actividades de Evaluación}
\end{center}}
\title{\cabecera}
\author{}
\date{}
%-------------------------------------------------------------

\newcounter{actividad}
\newcommand{\actividad}[1]{
  \stepcounter{actividad}
  \paragraph*{Ejercicio \theactividad. #1}
  }

\newcommand{\Vector}[1]{\ensuremath{\vec{\mathbf{#1}}}}
\renewcommand{\uu}{\Vector{u}}
\renewcommand{\nn}{\Vector{n}}
\newcommand{\rot}{\grad\times}
\renewcommand{\dx}[1]{\frac{\partial #1}{\partial x}}
\renewcommand{\dy}[1]{\frac{\partial #1}{\partial y}}

%-------------------------------------------------------------

\begin{document}

\maketitle

\begin{quotation}\small
  Los siguientes ejercicios se proponen como parte de la evaluación de
  la teoría de Dinámica de Fluidos Computacional (CFD).  El objetivo
  último es ayudar a entender y afianzar los conceptos expuestos en
  clase. Estos ejercicios se enfocan a tal efecto y por ello se ha
  optado por prolongar hasta el día anterior la fecha del examen el
  plazo para entregar estas actividades.  Deberán subirse a través de
  la tarea que está dispuesta a tal efecto en la sección ``CFD'' del
  campus virtual.
\end{quotation}

\begin{figure}
  \centering
  \includegraphics[width=0.25\linewidth]{dominio-circ-rect}
  \caption{Dominio $\Omega$ definido por una frontera circular,
     $\Gamma_1$, que contiene a otra frontera rectangular, $\Gamma_2$.}
  \label{dominio1}
\end{figure}

\actividad{} Sea el dominio $\Omega\subset\Rset^2$
(figura~\ref{dominio1}) definido por círculo con frontera
$\Gamma_1$ que contiene a un rectángulo de frontera
$\Gamma_2$. Consideramos un flujo con velocidad dada por
$$
\uu=\begin{pmatrix}
  \frac{\partial\Psi}{\partial x}
  \\[0.5em]
  \frac{\partial\Psi}{\partial y}
\end{pmatrix},
$$
siendo $\Psi:\mathbb R^2 \to \mathbb R$ la solución del siguiente problema:
\begin{equation}
  %\left\{
  \begin{cases}
    -\Delta\psi &=0 \quad\text{en } \Omega,
    \\
    \psi(x,y)&=g(x,y) \quad\forall (x,y)\in\Gamma_1,
    \\
    \psi(x,y)&=0 \quad\forall (x,y)\in\Gamma_2,
  \end{cases}
  \label{eq:psi.dirichlet}
\end{equation}
donde la función $g:\Gamma_2\to\Rset$ es una función dada.

\begin{enumerate}

\item ¿De qué tipo son las condiciones de contorno enunciadas en el
  caso anterior? ¿Cómo se denominan en el caso en que $g=0$? ¿Qué otro
  tipo de condiciones de contorno conoces?
\item Comprobar que se verifican las siguientes propiedades: el flujo anterior es
  incompresible ($\div\uu=0$) e irrotacional
  ($\rot\uu=0$)\footnote{Para el este flujo 2D, $\uu=(u_1,u_2)$, se
    define el rotacional de $u$ como
    $\rot\uu=\frac{\partial u_2}{\partial x} - \frac{\partial
      u_2}{\partial y}$.}.


\item Escribir la formulación variacional del
  problema~\eqref{eq:psi.dirichlet}, para el dato de contorno
  $g(x,y)=y^2$.
  \begin{quotation}
  \begin{footnotesize}
  \begin{emph}
    \textbf{Observación}: Para la formulación variacional, se recuerda la definición de los siguientes
    espacios de funciones, que fueron introducidos en las clases
    teóricas:
  \begin{align*}
    L^2(\Omega)&=\{ v:\Omega\to\Rset\ \text{tales que}\ \int_\Omega v^2 < +\infty\}
                 \ \text{(funciones de cuadrado integrable en $\Omega$)},
    \\
    H^1(\Omega)&=\{ v:\Omega\to\Rset\ \text{tales que}\ v, \dx{v}, \dy v \in L^2(\Omega) \}
                  \ \text{(funciones derivables en un sentido débil)},
    \\
    H^1_0(\Omega)&=\{ v:\in H^1(\Omega) \ \text{tales que}\ v=0 \text { sobre } \partial\Omega=\Gamma_1\cup\Gamma_2\}
                   \ \text{(funciones de $H^1(\Omega)$ que se anulan en la frontera}.
  \end{align*}
\end{emph}
\end{footnotesize}
\end{quotation}
\vspace{-2em}
\item Consideramos la aproximación del
  problema~\eqref{eq:psi.dirichlet} mediante elementos finitos
  ${\cal P}_1$. (a) Describir brevemente cómo se definen los espacios
  polinómicos a trozos para aproximar a solución de este problema.
  (b) La aproximación mediante elementos finitos, se obtiene como
  solución de formulación variacional que ha sido introducida en el
  apartado anterior? Justifica brevemente tu respuesta.


\item Escribir la formulación variacional del siguiente problema, en el que
  introducimos una condición de contorno de deslizamiento\footnote{
    Aquí $\nn(x,y)$ es el vector normal (exterior unitario) en cada
    punto $(x,y)$ de $\Gamma_2$. Por tanto, la condición de contorno
    anterior expresa que el flujo (el gradiente) es perpendicular al
    vector normal y, por tanto, paralelo a la frontera}:
\begin{equation*}
  %\left\{
  \begin{cases}
    -\Delta\psi &=0 \quad\text{en } \Omega,
    \\
    \psi(x,y)&=g(x,y) \quad\forall (x,y)\in\Gamma_1,
    \\
    \grad\psi(x,y)\cdot\nn(x,y)&=0 \quad\forall (x,y)\in\Gamma_2,
  \end{cases}
\end{equation*}
\end{enumerate}


\actividad{} Consideremos un flujo de Stokes el dominio $\Omega$ que
fue definido en el ejercicio anterior (figura~\ref{dominio1}):
  hallar $u=u(\xx)$, $v=v(\xx)$
    $p=p(\xx)$, todas ellas funciones definidas en un conjunto
    abierto $\Omega$ y con valores en $\Rset$, tales que:
    \begin{align}
      &- \nu\Delta u + \dx p = f_1
      \\
      &- \nu\Delta v + \dy p = f_2
      \div{\uu} &= \dx u + \dy v  = 0,
    \end{align}

\end{document}
